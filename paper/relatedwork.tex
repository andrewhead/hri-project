\section{Related Work}

\subsection{Optimization to Match User Preferences}

This paper considers algorithms that help a user find an ideal configuration with a small number of tests.
Methods to serve this purpose have been around for a long time.
Some early attempts at this problem are in the domain of optimal experiment design~\cite{box_empirical_1987}.
Optimal experiment design aims to reduce the number of scientific tests necessary to develop an empirical model of a phenomenon.
Because of the high cost of experiments, techniques including steepest ascent have been proposed to help researchers rapidly converge on an input space of interest~\cite{box_empirical_1987}.

Appropriate techniques for optimizing unknown, costly objective functions have been proposed in many additional directions.
These include the Nelder-Mead method~\cite{nelder_simplex_1965},
% The Nelder-Mead method searches for a maximum by querying the scores at the vertices of a simplex in the input space and adjusting these vertices in direction of maximization~\cite{nelder_simplex_1965}.
evolutionary strategies (e.g., CMA-ES~\cite{hansen_adapting_1996}),
Bayesian optimization~\cite{brochu_tutorial_2010},
and reinforcement learning~\cite{sutton_reinforcement_1998}.
Active learning, similar to optimal experiment design, aims to learn a model with a small number of strategically-chosen samples~\cite{cohn_active_1996,settles_active_2010}.
% We draw upon the principles for successful determination models from the generic techniques from the body of literature on active learning~\cite{cohn_active_1996,settles_active_2010}.

% An overarching theme from this body of work I use to frame the results is ``exploration'' vs.\ ``exploitation''~\cite{sutton_reinforcement_1998}.
% This means that a large enough portion of the function's domain is covered to be reasonably certain that optima found are global, not local;
% at the same time, the algorithm does not spend too long exploring regions that are not likely to contain the global optimum.

This work ultimately aims to lessen the effort to train a machine to learn a human-provided goal.
In this aim, it is similar to past work on robotic ``clicker training''~\cite{kaplan_robotic_2002}~\cite{grollman_dogged_2007}.
Clicker training aims to teach robots complex behavior with a single binary input channel that can reward partial behaviors until a robot builds up the full behavior.
This is a modified version of reinforcement learning, where a human judge offers rewards to a robot so that it can learn an ideal policy of how to act when it's in a certain state.
% More concretely for this instantiation of this project, I am interested in modeling user preference.
% A problem framing of Bayesian optimization for user preference modeling that motivated this work can be found in~\cite{brochu_tutorial_2010}.

This work builds on past reviews of optimization techniques (e.g.~\cite{hansen_comparing_2010,mersmann_benchmarking_2010}).
Rather than focusing on performance in this paper, we focus on human perception of algorithms, with an additional discussion of performance.

\subsection{Interactive Fabrication}

User activities in this paper focus on fabrication machines.
This is due to the progressively lower cost of these machines, and predictions that they may someday become home and workshop appliances.
A body of peripherally related work has focused on improving peoples' ability to express their design intentions with fabrication machines.
In these works, the machine collects a user's intent based on models and sketches they provide and modifies a workpiece as a result.

FreeD~\cite{zoran_human-computer_2013,zoran_hybrid_2014} is a semi-autonomous tool for carving shapes in 3D.
A user provides a virtual 3D model to the machine as their goal.
As they work with an uncarved workpiece, a user makes large-scale motions.
The milling device changes its own speed to remove more or less material, based on its position relative to the virtual 3D model.
The authors' take effort to minimize fabrication risk while allowing authentic engagement with raw material~\cite{zoran_hybrid_2014}.
\emph{Constructable}~\cite{mueller_interactive_2012} augments a laser cutter, aiming to enable interactive construction, where users interact by sketching on a workpiece with a laser pointer.
The system observes laser pointer motions and converts them into constraint sets for actions, which it performs on the workpiece~\cite{mueller_interactive_2012,mueller_laserorigami_2013}.
My work, in contrast to past work, enables user to achieve some optimal goal with a workpiece, rather than enabling creative engagement with material or speeding up the expression of toolpaths.

% Other systems (e.g., ~\cite{mueller_wireprint_2014}~\cite{savage_sauron_2013}), provide mechanisms for rapidly prototyping both non-interactive and interactive workpieces.
% This is with an intent to speed up the design process.
