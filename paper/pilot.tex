\section{A Preliminary Lab Study}

Before the experiment above, I ran an in-lab study with three participants.
As the motivating domain for this paper is fabrication, I sought to compare one optimization technique---Nelder-Mead---with a baseline interface for discovering the right engraving settings.
In particular, participants would view and rank physical, laser-engraved examples.

Participants used either the Nelder-Mead interface or a slider-based interface to guide the laser cutter to choosing engraving parameters.
The slider-based interface was a near-direct copy of the software user interface for the Universal Laser Systems VersaLaser 3.50, a modern laser cutter.
Just like in the original interface, there was no indication of how these parameters would affect how the laser engraved, or how the result would appear.
The experimenter provided no description of the parameters.

Similarly to the MTurk experiment, participants were provided with a goal---
a physical 2cm$\times$2cm tile with an engraved `O'.
This was cut on a soft particle wood 3mm thick.
With the slider-based interface, participants moved sliders to change the values of power, speed and PPI to alter the engraving appearance.
They were given pen and paper to take notes as they worked to understand how these parameters affected the appearance.
With the Nelder-Mead interface, similar to above, participants ordered engravings from left to right based on their closeness to the goal.
Unlike the MTurk study, participants in the lab could actually do this ranking with the physical workpieces.
After ordering them on the table, they entered their ranking into the software interface.
The experimenter fed the suggested parameters from the algorithm into the laser cutter to engrave each new example.

While participants' screens were recorded and all examples saved, I have yet to process this data.
My intuitions are that participants were able to try out more examples with less time in-between with the Nelder-Mead interface than with the baseline interface.
However, participants were also frustrated, confused, or skeptical when the Nelder-Mead interface converged on an maximum that was not their goal.
Feedback from this study also inspired the author to improve the ranking interface to support insertion sort-based ranking mechanics, to avoid showing a reflection, expansions and contraction all at once, and to improve bugs that were uncovered.
