\section{Algorithms}

I implement two algorithms that enable the discovery of user preferences.
Here, I provide a brief introduction to each of the algorithms.
I describe the steps taken to modify the algorithms to accept comparisons as input to drive optimization.
I also describe how bounds are enforced for each one, the initial points of each algorithm, and the hyperparameters used.

The input space for each algorithm was discretized, with a log-scale for each of three dimensions: the laser cutter's settings for power, speed, and PPI\@.
Power and speed ranged from 1--100\%.
Possible values for these two dimensions were:
$\left[1\%, 3\%, 10\%, 32\%, 100\%\right]$.
PPI ranged within 10--1000, with possible values:
$\left[10, 32, 100, 316, 1000\right]$.
In the space below, configurations are described as three-tuples:
$\left(Power, Speed, PPI\right)$.
For a discussion of how each of these parameters impact the laser cutter's operation, see~\cite{versalaser_2010}.

\subsection{Nelder-Mead Optimization}

Nelder-Mead is an optimization method that iteratively moves vertices of a simplex towards the best-rated vertex~\cite{nelder_simplex_1965}.
For reflection, expansion, and contraction, we chose coefficients observed as the best in the original Nelder-Mead work~\cite{nelder_simplex_1965}:
$\alpha=1$,
$\gamma=2$,
$\beta=-1/2$,
and a reduction coefficient of 1/2.

To perform Nelder-Mead optimization, the algorithm doesn't need to know an exact evaluation for each vertex.
It only needs to know which is the best, second-to-worst, and worst points out of each set of vertices.

Initially, the vertices were:
(1\%, 1\%, 10),
(1\%, 1\%, 1000),
(1\%, 100\%, 10),
(100\%, 1\%, 10),
These points were chosen for two reasons.
First, at least four points are needed to migrate to all possible points in 3D space.
(With any three points, all three and their centroid would reside on a single plane.
Any reflection, expansion, contraction computed as a transformation of a difference vector on that plane yields another point on that plane.)
Second, it was clear from viewing this collection of points that extremes of appearance were shown.
For example, one image consisted of very sparse dots, another of very light lines, and another showed dark, thick lines.

\subsection{Bayesian Optimization}

Each iteration of Bayesian optimization comprises two steps~\cite{brochu_tutorial_2010}:
fitting a Gaussian process to the data seen so far, and sampling a new value likely to maximize the unknown cost function.
Typically, selection of the sample in the second step is performed by maximizing \emph{expected improvement}.
Based on the work by Brochu et al.~\cite{brochu_tutorial_2010}, I express this as combination of two terms for exploration and exploitation:
\begin{equation}
EI (x) = (\mu(x) - f (x^+)) \Phi(Z) + \sigma(x) \phi(Z)
\end{equation}
where $Z = \frac{\mu(x) - f(x^+)}{\sigma(x)}$ is in the domain of a normal distribution, representing the probability of improvement at $x$;
$\mu(x)$ is the predicted value of a new input $x$;
$f(x^+)$ is highest value seen so far;
$\sigma(x)$ is the standard deviation at $x$;
$\phi(Z)$ and $\Phi(Z)$ are the probability and cumulative density functions for $Z$.

Brochu et al.~\cite{brochu_tutorial_2010}~\cite{brochu_active_2008} propose a variant on Bayesian optimization that takes comparisons between sampled points as inputs.
I implement this algorithm for this study.
I used a squared exponential kernel with $\sigma = 0.25$.
In performing Newton-Rhapson optimization according to Brochu et al.'s formulation, $\sigma_{noise}$ was set to 10, and ten iterations were performed to optimize the Gaussian process.
These parameters were chosen by trial and error, observing what appeared to enable convergence in the target 3D input space for this study.
Caching and limited iterations were implemented to ensure the algorithm would update to reflect user input in around one second, even after twenty comparisons had been given.

The algorithm was seeded with one comparison:
(3\%, 3\%, 32) and
(32\%, 32\%, 316).
These two points were chosen to cover two distinct ends of the input space with distinct appearances.

In addition to the static bounds, the algorithm was restricted to select examples where the material had not burned during cutting.
`O's that had fallen out of the material, or incomplete jobs that were aborted due to flames, could not be chosen by the acquisition function.
I expected that users viewing these samples would be confused as to why the jobs weren't complete and provide comparisons that did not fit nicely to a smooth model of user preference based on the input space.
